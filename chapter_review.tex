\chapter{Reviewing Other Implementations}

The first question any responsible engineer (or lazy hobbyist) asks
themself is: ``Has someone already solved this?''  After a bit of
googling, I found a couple of existing AVR \aprs /Packet Radio
projects and had a look at them.

\section{A Philosophical Diversion}

(Skip this section if you're not interested in the philosophical take
behind my kit-making business.  It's really only of interest to other
kit makers, but I leave it here out of selfish conceit.)

The goal of my project is to make an AVR-based TNC that can be easily
duplicated by others, and may or may not make a compelling kit for my
company to sell.  With that in mind, there are a few principles that
are worth mentioning:

\begin{enumerate}
\item Hackability: the device should be transparent and open-ended.
\item Free/Open Source license: this is going to be Open Source Hardware,
  and I want the code to be Open Source as well (ideally GPL).
\item Beauty: because aesthetics always matter
\item Approachable: people should be able to learn how the device
  works, even if they're totally new
\end{enumerate}

So, as I looked at other projects, I had a few criteria in mind for
their acceptability.  Some of these are rules of thumb, but most of
them are there to enforce one of the principles above.

\begin{enumerate}
\item Component count: the fewer parts the better.
\begin{itemize}
\item Hackability: Fewer components leads to simpler designs; simpler
  designs are easier to extend.
\item Approachability: Simpler designs are easier to explain to people
  who don't know a whole lot.  They may not be able to extend them
  from their knowledge, but they're going to feel a better sense of
  ownership, and be more motivated to learn more.
\end{itemize}

\item Simple algorithms/structures: this is not a complicated problem; using
  complicated techniques is probably not needed.
\begin{itemize}
\item Hackability: This is the software equivalent of the component
  count criterion.
\item Approachability: Ditto.
\end{itemize}

\item Understandable code: the world needs more beautiful
  microcontroller code.
\begin{itemize}
\item Beauty: This is kind of self-explanatory, right?
\item Hackability: There is very little more frustrating than picking
  your way through someone's awful spaghetti code to make it do
  something slightly different.  Most of the microcontroller code I've
  seen on the web is bad.  As in, it's the least-approachable code
  I've seen outside of academia.  We can do better as a community.
\item Approachability: Beautiful code makes microcontrollers a lot
  more approachable and usable by the programmer community.  A large
  part of learning to code is cutting and pasting someone else's code,
  then experimenting with various tweaks.  Well-written code is
  amenable to changes by people without a mental model of the whole
  system.
\end{itemize}

\item Frugal code: if there's no CPU left for people to hook in new
  features, this is a closed-ended device.
\begin{itemize}
\item Beauty: the most beautiful code is often the shortest code.  The
  shortest code is often the fastest code.
\item Hackability: if someone wants to use this device as a platform,
  they need to have some CPU to spare to do so.  It's just evil to
  make them optimize our code for us, so they can fit theirs in.
\item Approachability: new embedded coders are, shall we say,
  ``relaxed'' about wasting CPU cycles or memory on bad
  algorithms.  Instead of forcing them to climb the very steep
  learning curve before they've had any success, we should endeavor to
  give them as much opportunity to succeed as we can.
\end{itemize}

\end{enumerate}


\section{Evaluation Criteria}

So, with all that out of the way, here are the criteria I used to
evaluate the existing implementations I found:
\begin{enumerate}
\item \label{criteria:components} How many components does it need to?
\item \label{criteria:frugality} How frugal is the code, in both time
  and space?
\item \label{criteria:approachability} How approachable is the software?
\item \label{criteria:license} Is it available in an open-source license?
\end{enumerate}


\section{Features Desired}

There are a handful of features the device should have:

\begin{itemize}
\item Full KISS-mode TNC: KISS is a TNC-control protocol\cite{KISS}
\item TTL-level serial interface; probably with a designed-in spot for
  RS-232 and DB-9 (DTE pinout, on-board jumpers for DCE/DTE?)
\item Audio modulation for packet transmission, respecting DCD
\item Audio demodulation for packet reception
\item Push-to-talk button support
\end{itemize}

\section{WhereAVR}

WhereAVR\cite{WhereAVR} is a project by Gary Dion (N4TXI) that
implements a fairly clean design, with a lot of nice features.  It is
actually a fairly nice design.  It has a modulator and demodulator, as
well as a GPS parsing interface within it.  

\subsection{Missing Features}

Before getting ahead of ourselves, let's be clear: WhereAVR doesn't
actually solve the problem at hand.  That is: it's not a packet TNC,
it's ``just'' an \aprs transmitter.  It implements a receive function,
but doesn't actually do anything with it, besides light up a DCD
light.  That said, this could be a very good starting point to extend
out.

\subsection{Component Count}

WhereAVR does middling-well on component count.  Its biggest gotcha is
using a switch-mode (boost) power supply, which is really unfortunate
for newbies.  Happily, this is ancillary to the rest of the design, so
it can simply be skipped over if desired.

The GPS interface isn't a good fit for our application, either: using
a 2N3904 level inverter for RS-232 -> TTL is a longstanding tradition,
but it's a more to learn for newbies than a MAX232 ``Magic converter
chip'' would be\footnote{That said, this could be a good opportunity
  to introduce people to using BJTs as switches.}.   Again, this is
something that can just be cut out.

The audio output is very nice: it uses a 4 bit R/2R DAC to generate
waveforms.  This is fed into a trimpot for volume control, and
capactively coupled to the microphone jack.  All in all: good stuff.

\subsection{Code Frugality}

This code is pretty good about its resource utilization.  Lots of
things are done via interrupts, which makes it a little difficult to
ascertain exactly how much CPU is left after the APRS overhead, but it
looks to be pretty good.  The memory footprint is also very good by
today's standards: it was written for the ATMega8, which has a quarter
the resources of the now-standard ATMega328!

In particular, his use of zero-crossing interrupts to measure input
frequency is very nice.  That said, the algorithm misses a class of
important corner case, which causes packet loss.

\subsection{Code Readability}

This is where WhereAVR shines.  It's really well-commented,
well-organized, and, most importantly, very consistent.  It really
feels like Dion went out of his way to keep the indirect linkages to a
minimum and the structure clear.


\subsection{License}

Unfortunately, WhereAVR is released under a very restrictive license
(but one very common in the Amateur Radio community).
\begin{quote}
  This software is available only for non-commercial amateur radio or
  educational applications.  All other uses are prohibited.  This
  software may be modified only if the resulting code be made
  available publicly and the original author(s) given credit.
\end{quote}

This is, of course, wholly unacceptable for our desired application.

\section{Arduino TNC}
\subsection{Missing Features}

Arduino TNC\cite{ArduinoTNC} is very nearly feature-complete: it has a
KISS implementation, transmit and receive functionality, and, as an
Arduino, has TTL serial built in.  It doesn't have support for a PTT
button, but with VOX transmission, that's not really necessary.

\subsection{Component Count}

This is delightfully light on the components: short and to the point.
It uses the same transmit and receive scehmatics as WhereAVR, bummed
down to the minimal circuit.  This is fantastic.

\subsection{Code Frugality}

It's very difficult to tell how frugal this code is.  It uses two
timers, a lot of global variables, and fairly sizeable packet buffer.
It comes in at somewhere around half a kilobyte of RAM, and looks like
it will use a fair bit of CPU.  Most importantly, all of the receive
functionality is implemented within the \texttt{ TIMER1\_OVF\_vect }
handler, with interrupts disabled (and is over 1,000 instructions
long, firing every 1,200 instructions).  This makes it very difficult
for auxillary processing to coexist with the code.

A lot of this overhead comes from the receive functionality, which is
very CPU heavy.  It is based on a technique documented in Cypress
Application Note 2336\cite{cypress2336},  which is for a specialized
Cypress Programmable System-on-Chip.  These have built-in manipulable
analog and digital components; something like an FPGA for mixed signal
processing.  This is, of course, a far cry from the capabilities of
the Atmega328.

\subsection{Code Readability}

This code is written in the older school style: nearly all the
functions have a type signature of void f(void).  All state is shared
via globals, which are modified all over the place.  It has a lot of
helpful comments, but it's unlikely that one could simply jump in and
tweak any one part of it without having to worry about breaking other,
mostly-unrelated parts, in the process.


\subsection{License}

The license for this is never specified, but it lifts heavily from
WhereAVR, so it will technically fall under the same unfortunate
license.

\section{BeRTOS Example code}

BeRTOS\cite{BeRTOS} is a free, open-source real-time kernel and
support libraries.  It runs on multiple embedded systems, including
the AVR.  One of their example applications\cite{BeRTOSAPRS} is an APRS modem, based on
their libraries (which already have all the requisite functionality).

\subsection{Missing Features}

Their example is actually feature-complete: it includes transmit and
receive, has a PTT button, runs on an Arduino (TTL serial), and can
act as a KISS modem (in part thanks to Robert Marshall, the author of
Arduino TNC).


\subsection{Component Count}

The schematic for this is another copy of the WhereAVR schematic, with
the addition of a simple transistor to handle the PTT button.  It's
clearly laid out on their site, and looks to be pretty solid.

\subsection{Code Frugality}

BeRTOS is a whole real-time operating system, with a lot of built-in
libraries, for multiple platforms.  That is to say: their code is
pretty good, but it's also not exactly minimalist.

The BeRTOS AFSK1200 modulator uses a 128 byte quarter-sine table, in
contrast with the 16 byte ones used in WhereAVR and Arduino TNC.

Their demodulation method is inappropriate for our uses.
It's advanced, and is probably more robust than any other technique
that's implemented on microcontrollers, but it achieves this by
implementing digital versions of analog signal filters\footnote{They
  have a whole IIR implementation of Chebyshev and Butterworth
  filters; for all the gory underpinnings of these, see their code,
  the obvious Wikipedia pages, and Hamming's \textit{Digital
    Filters}\cite{HammingDigitalFilters} or Oppenheim and Shafer's
  \textit{Digital SIgnal Processing}\cite{OppenhiemSchaferDSP}.}.

This is CPU and memory intensive: the receive ISR is nearly five
hundred instructions long, called every 1700 clocks.  This isn't as
bad as others, but is still problematic.

\subsection{Code Readability}

BeRTOS is a mature, professionally-produced product, and the code
quality reflects this.  Everything is neat and orderly, with good
comments and encapsulated variables.  It's a little intimidating
because it's a full operating system, but it's quite readable code.

That said, reusing it is tricky: it appears that, much like Arduino
itself, one needs to fully ``drink the kool-aid'' and develop a BeRTOS
application to build on top of this.  It's impossible to simply
cut-and-paste their code into a new project and have it compile.
Instead, a whole environment has to be created.  However, they have
tools which make this easier, and it comes with many nice libraries.

\subsection{License}

BeRTOS is released under the GPLv2 (or later), which is perfect for
our application.  They also have a GPL exception for simply repackaging
their library, which is very nice: you may use BeRTOS in closed
products so long as you don't modify it and you do acknowledge their
contribution.



\section{Other Implementations Worth Noting}

\subsection{multimon}

multimon\cite{multimon} is a software modem that works with audio data taken from the
soundcard (``soundmodem'').  It runs on a Linux PC, which is clearly
not the target platform, but it does provide a platform to test
against.  As development progressed, multimon proved to be a very
useful tool for decoding data.

\subsection{OpenTracker+}

The OpenTracker+\cite{opentracker} is the most popular open source
\aprs tracker out there today.  It uses the HC08 processor, with code
distributed as a Metroworks CodeWarrior project.  This is something
everyone can read, but the code is mostly assembler.

\section{Conclusions}

The closest fit for our desired features is the BeRTOS example code.
It's got a very solid design and clean code, which would make a fine
basis for our application.  Unfortunately, it uses too many
microcontroller resources to really fit.  This means that we're faced
with implementing things ourselves.

With that in mind, let's at least sketch out an implementation of the
subsystems.  We will identify which parts of the above codebases we
can learn from, and, most importantly, how we can test our subsystems
as they're implemented.

\subsection{AX.25/HDLC Implementation}

AX.25 and HDLC are hard to separate: they're a great example of a
leaky abstraction \footnote{If this is a new term to you, check out
  Joel Spolsky's excellent \textit{The Law of Leaky Abstractions}
  \cite{LeakyAbstractions}.  It's a good read, and worth meditating on
  and/or chuckling over as you implement microcontroller code.}.  This
makes it a little rough to test them independently.  For our purposes,
we'll tease out AX.25, but HDLC will wind up being a very thin layer
on top of this.

The best example of AX.25 encoding is found in WhereAVR, though it's a
bit gnarly: it outputs by directly manipulating the timer state
values.  The best decoding example is in BeRTOS; note that
\texttt{kfile()} is their form of a pipe between processes.

It may also be helpful to tweak multimon's \texttt{gen\_hdlc.c}
routines to dump state to a text file, then build a teststand around
those files.  It has the benefit of being designed to run on a PC from
the beginning, instead of on a microcontroller.

One nice property of this: you can build both sides simultaneously and
then make sure they work with each other, before moving on to a
bake-off with other implementations.  This is helpful, insofar as it
forces you to clarify your own understanding of the protocols.


\subsection{AFSK1200 Modulation}

The mechanisms used in WhereAVR feel like a good fit for our needs.
That said, I would like to simplify a bit from what he's got.  There
are alternative methods for generating the sine wave, such as using
PWM through a low-pass filter.  This is a tried and true method, and
is well documented.  The best overall overview is \textit{Filtering
  PWM Signals}\cite{WagnerPWM}, by Jim Wagner.  Just about every
microcontroller manufacturer has an application note on the subject,
as well.

Testing is fairly straightforward here: once we have output going,
we'll pipe it into multimon.  For this purpose, I would recommend a
cheap USB sound card.  There's also a lot to be said for spending a
lot of time with Audacity, staring at waveforms directly.  At one
point, I found myself able to decode AFSK by simply scrolling along a
waveform.

\subsection{AFSK1200 Demodulation}

For demodulation, the zero-crossing detector in WhereAVR is
delightfully simple, but doesn't feel robust enough against phase
drift.  That said, the more advanced ADC with digital filter
techniques used in Arduino TNC and BeRTOS are tempting.  It seems
reasonable to step back and see what can be done to condition the
output of a zero-crossing detector, and how to make it handle phase
error more gracefully.

\subsection{KISS Interface}

The KISS interface in ArduinoTNC is actually pretty clean\footnote{
  It's in the function named \texttt{MsgHandler()}, if that wasn't
  immediately obvious.}, and is worth basing things off of.  To
generate test data, there is also very nice test code by Michael
Toren, implementing a KISS sender\cite{BacchusKISS}.
